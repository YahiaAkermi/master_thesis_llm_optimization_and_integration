% Introduction
\chapter{Introduction}
Blockchain technology has gained increasing attention in both industry and academia due to its potential to transform digital systems~\cite{state2016blockchain}. At its core, blockchain is a decentralized public ledger where transactions are grouped into blocks and linked chronologically to form an immutable chain~\cite{nakamoto2008bitcoin}. This structure enables the secure recording of transactions without relying on a central authority.

To ensure security and consistency across the network, blockchain systems incorporate asymmetric cryptography and distributed consensus algorithms. These elements provide critical features such as decentralization, immutability, anonymity, and auditability, which contribute to lower operational costs and improved system efficiency~\cite{nakamoto2008bitcoin}.

These qualities have enabled blockchain to support a wide array of applications beyond financial transactions. One of the most notable uses is in smart contracts—self-executing agreements with terms encoded directly into code. Once deployed on a blockchain, these contracts are executed automatically by the network (e.g., by miners), ensuring trustless, tamper-proof enforcement of conditions~\cite{kosba2016hawk}.

In addition to smart contracts, blockchain is increasingly applied in public services, Internet of Things (IoT) environments\cite{zhang2015iot}, reputation systems\cite{sharples2015blockchain}, and security-related services~\cite{hyperledger2015}. In these domains, blockchain’s immutability assures data integrity, while its distributed nature eliminates single points of failure, making systems more resilient and transparent.

However, despite its advantages, blockchain faces several technical challenges. Scalability is a primary concern, as block size limitations restrict the number of transactions processed per second. This leads to trade-offs between performance, storage, and decentralization~\cite{nakamoto2008bitcoin}. Moreover, issues such as privacy leakage have been identified even when users transact using cryptographic keys alone~\cite{biryukov2014deanonymisation}. The consensus mechanisms themselves also introduce drawbacks: proof of work consumes significant energy, while proof of stake can reinforce economic inequality among participants~\cite{biryukov2014deanonymisation}.

In conclusion, blockchain represents a powerful infrastructure for decentralized data management and automation through smart contracts. Yet, to unlock its full capabilities, ongoing research and innovation are required to address its current limitations.

\chapter{Challenges}
Smart contracts, while foundational to decentralized applications, present several inherent security challenges due to their immutable and autonomous nature. Once deployed, their code cannot be modified, meaning any vulnerability becomes permanently accessible on the blockchain. This immutability, combined with full transparency, allows attackers to analyze contract logic in detail and identify potential exploits without restriction. Many smart contracts encapsulate complex interactions and business logic, which can lead to critical errors such as reentrancy bugs, arithmetic overflows, and improper exception handling. These issues are often exacerbated by the lack of standardized development practices and the relative immaturity of tooling in the ecosystem. Furthermore, contracts frequently rely on external dependencies, including other smart contracts and off-chain data provided by oracles—each introducing additional risks that may not be fully under the developer's control. Despite efforts to improve testing and auditing tools, formal verification remains underutilized, and manual reviews are time-consuming and error-prone. As a result, even small flaws can lead to severe financial losses, making the identification and prevention of vulnerabilities a critical aspect of secure smart contract development.

\chapter{Motivation}
The increasing reliance on smart contracts for critical operations in finance, governance, and supply chain management introduces a growing demand for secure and dependable execution. However, the irreversible nature of smart contract deployment, combined with the complexity of their logic and the absence of centralized oversight, makes them highly susceptible to costly vulnerabilities. As demonstrated by numerous real-world exploits, even minor coding flaws can lead to severe financial and reputational consequences.

This thesis is driven by the need to develop a more robust and automated approach to vulnerability detection in smart contracts. Traditional manual auditing and static analysis methods, while useful, often fall short in scalability and accuracy. The integration of advanced deep learning techniques offers a promising path forward—enabling the automatic discovery of subtle and complex vulnerability patterns that may be overlooked by conventional tools. The aim of this thesis is to present a comprehensive state-of-the-art review of how deep learning techniques are being applied to the detection of vulnerabilities in smart contracts. By analyzing current research and approaches, this work seeks to provide a clear understanding of the field’s progress, challenges, and future potential. The motivation behind this study is to contribute to the development of a more secure and resilient smart contract ecosystem through intelligent, automated analysis methods.

\chapter{Organisation And Structure}
This document constitutes the first phase of an in-depth research project focused on the detection of vulnerabilities in smart contracts using deep learning techniques. It is structured into five main chapters, organized as follows:

\begin{itemize}
    \item \textbf{Chapter 1: Introduction} \\
    This chapter introduces the research topic, outlines the motivation, defines the problem statement, and presents the main objectives and scope of the study.

    \item \textbf{Chapter 2: Background} \\
    This chapter provides the necessary theoretical foundation by covering two key areas: blockchain technology and smart contracts on one hand, and deep learning principles and architectures on the other. It offers the reader a comprehensive understanding of the technical context surrounding both domains.

    \item \textbf{Chapter 3: Related Work} \\
    This chapter reviews existing approaches and tools for detecting vulnerabilities in smart contracts. It examines traditional analysis techniques as well as recent research leveraging machine learning and deep learning, highlighting their strengths and limitations.


    \item \textbf{Chapter 4: Conclusion} \\
    The final chapter summarizes the key contributions of the research, reflects on its limitations, and suggests potential directions for future work.
\end{itemize}

The document concludes with a complete list of references used throughout the thesis.